\chapter{DUNE long baseline analysis}
\label{sec:dune_lbl}

%%% Overall point of this chapter: How do our uncertainties relate to our ability to measure CPV
As covered in \citechap{sec:dune}, DUNE's primary goal is to make a measurement of CP violation (CPV) in the lepton sector.
Prior to any construction of the experiment, it is important to assess DUNE's eventual sensitivity to both CPV and other oscillation physics, such as the neutrino mass ordering.

\section{Summary of DUNE systematic uncertainties}
\label{sec:dune_lbl:systs}

This section gives a summary of the systematic uncertainties used in DUNE's long baseline analysis and where they are derived from.
\citesec{sec:dune_lbl:systs:flux} details the systematics deriving from lack of knowledge about the neutrino beam (commonly referred to as the `flux' systematics).
\citesec{sec:dune_lbl:systs:xsec} gives details of those systematics relating to lack of knowledge surrounding neutrino interactions with matter.
Finally, \citesec{sec:dune_lbl:systs:det} outlines the systematic uncertainties which derive from the uncertainty surrounding detector effects, for example the energy scale of the detector.

\subsection{Flux}
\label{sec:dune_lbl:systs:flux}

\subsection{Neutrino interaction (cross-section)}
\label{sec:dune_lbl:systs:xsec}

\subsection{Detector}
\label{sec:dune_lbl:systs:det}