\chapter{DUNE long baseline analysis}
\label{sec:dune_lbl}

%%% Overall point of this chapter: How do our uncertainties relate to our ability to measure CPV
As covered in \citechap{sec:dune}, DUNE's primary goal is to make a measurement of CP violation (CPV) in the lepton sector.
Prior to any construction of the experiment, it is important to assess DUNE's eventual sensitivity to both CPV and other oscillation physics, such as the neutrino mass ordering.

\section{Summary of DUNE systematic uncertainties}
\label{sec:dune_lbl:systs}

This section gives a summary of the systematic uncertainties used in DUNE's long baseline analysis and where they are derived from.
\citesec{sec:dune_lbl:systs:flux} details the systematics deriving from lack of knowledge about the neutrino beam (commonly referred to as the `flux' systematics).
\citesec{sec:dune_lbl:systs:xsec} gives details of those systematics relating to lack of knowledge surrounding neutrino interactions with matter.
Finally, \citesec{sec:dune_lbl:systs:det} outlines the systematic uncertainties which derive from the uncertainty surrounding detector effects, for example the energy scale of the detector.

\subsection{Flux}
\label{sec:dune_lbl:systs:flux}

Uncertainties in the neutrino beam flux primarily arise from two sources: uncertainties in the production of hadrons by protons striking the beam target and uncertainties in the design parameters of the beam such as the horn currents or positioning (commonly referred to as the ``focussing uncertainties''.

The focussing uncertainties are evaluated by varying the beamline parameters within their tolerances and observing the resulting changes in the neutrino fluxes.
Hadron production uncertainties are estimated using uncertainties from thin target data experiments such as NA49~\cite{na49} with large uncertainties assigned to those interactions that are not covered by data.

\citefig{fig:fluxUncertainties} shows these uncertainties as a function of neutrino energy for the \numu and \anumu fluxes in both neutrino and antineutrino mode.
One can see that, at nearly all energies, hadron production uncertainties are larger than focussing uncertainties.

\begin{figure}[h]
  \centering
  \includegraphics[width=.8\linewidth]{files/figures/dune_detector/fluxUncertainties}
  \caption[Far detector flux uncertainties for \numu and \anumu]{Far detector \numu and \anumu flux uncertainties as a function of neutrino energy, from~\cite{tdrVol2}. The uncertainties are separated into those resulting from hadron production and those resulting from the focussing uncertainties.}
  \label{fig:fluxUncertainties}
\end{figure}

\subsection{Neutrino interaction (cross-section)}
\label{sec:dune_lbl:systs:xsec}

\subsection{Detector}
\label{sec:dune_lbl:systs:det}

Detector systematics in the DUNE long baseline analysis primarily take the form of bin-to-bin shifts, rather than the weights typically used for the flux and cross-section systematics.
For various particle types there exist multi-parameter energy scale shifts of the form
\begin{equation}
E_{\alpha,~\text{reco}}' = E_{\alpha,~\text{reco}} \left( p_{0} + p_{1} \sqrt{E_{\alpha,~\text{reco}}} + \frac{p_{2}}{\sqrt{E_{\alpha,~\text{reco}}}} \right)	
\end{equation}
where the $p_{n}$ are allowed to vary in any fits.
\citetab{tab:energyScaleParams} shows the $1\sigma$ values of the $p_{n}$ for each of the particle types.

\begin{table}
	\caption[$1\sigma$ uncertainties for the detector energy response used in the DUNE long baseline analysis]{$1\sigma$ uncertainties for the detector energy response used in the DUNE long baseline analysis. The muon curvature uncertainty only applies to those muons in the near detector which pass into the magnetised section of the detector.}
	\begin{tabular}{c c c c}
		\hline
		\hline
		Particle types & \multicolumn{3}{c}{$1\sigma$ variation} \\
		& $p_{0}$ & $p_{1}$ & $p_{2}$ \\
		\hline
		All, except muons      & 2\%   & 1\%   & 2\%   \\
		$\mu$ (range)          & 2\%   & 2\%   & 2\%   \\
		$\mu$ (curvature)      & 1\%   & 1\%   & 1\%   \\
		$p$, \pipm             & 5\%   & 5\%   & 5\%   \\
		$e$, $\gamma$, \pizero & 2.5\% & 2.5\% & 2.5\% \\
		$n$                    & 20\%  & 30\%  & 30\% \\
		\hline
	\end{tabular}
\end{table}
