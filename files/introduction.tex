\chapter{Introductory Material}
\label{sec:intro}

%%% Explanations of what is in each chapter
\citechap{sec:theory} provides the theoretical background for much of this thesis.
We begin with a brief history of discoveries in the neutrino sector to provide
In particular, a theoretical overview of the neutrino, its interactions and the phenomenon of neutrino oscillations is given.
The chapter concludes by presenting the current experimental status of neutrino oscillations and prospects for the future.

\citechap{sec:dune} gives an overview of the DUNE experiment. 
This includes a description of the physics goals of the experiment, the technologies used in the detectors. 
A brief summary, including key technical details, is provided for each part of the experiment.

\citechap{sec:dune_lbl} expands on the previous chapter by providing an assessment of the systematic uncertainties assumed by DUNE.
Also explored is the effect these uncertainties have on the sensitivity of DUNE to various physical parameters involved in neutrino oscillations.

\citechap{sec:dune_ndrwt} provides an insight into hpw the gaseous argon component of the DUNE near detector can be used to correct for lack of knowledge about how neutrinos interact with matter.
The analysis also provides a demonstration of how this information can be used to help DUNE meet its physics goals.

\citechap{sec:hptpc_dtof_characterisation} describes the construction and characterisation of a time of flight system. 

\citechap{sec:hptpc_beam_flux} gives details of an analysis involving this time of flight system which took place at CERN in 2018.
This analysis demonstrates a novel method of producing a low energy proton beam.
This is of relevance when making some measurements relating to neutrino interactions.

\citechap{sec:protodune} provides a brief overview of the ProtoDUNE-SP detector, a prototype detector located at CERN.
Within the chapter, details of each of the major components of the detector are given.

\citechap{sec:pdune_calibration} provides a method of measuring the electron lifetime within the ProtoDUNE-SP detector.
Measurements of this electron lifetime are then compared with those from another source to give an estimate of a possible calibration error.
This calibration error is then propagated to DUNE far detector simulations where its effect on the measurement of neutrino oscillation parameters is assessed.

Finally, \citechap{sec:conclusions} presents general conclusions drawn from the work in this document and provides an outlook for future work.