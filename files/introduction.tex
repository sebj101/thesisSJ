\chapter{Introductory Material}
\label{sec:intro}

%%% Big picture things
The Standard Model has provided remarkably accurate predictions of microscopic physical phenomena since it was first formulated in various parts during the 1960s and 1970s~\cite{glashow1961, weinberg1967, qcdTheory1, qcdTheory2}.
As a whole it provides a coherent description of three of the four fundamental forces of nature which has withstood every experimental test posed thus far. 
The crowning achievement of the theory came in 2012, with the discovery of the Higgs boson at CERN~\cite{higgsDiscoveryATLAS, higgsDiscoveryCMS} (as first predicted by several groups of theorists~\cite{higgsTheory1, higgsTheory2, higgsTheory3} in the 1960s).

%% Problems with the Standard Model
However, we know that the Standard Model cannot provide a complete description of reality.
Although it provides robust descriptions of electromagnetic, weak and strong interactions it cannot be reconciled with our current understanding of gravity at all scales~\cite{quantumGravity}.
It provides no widely accepted explanation for several observed astrophysical phenomena.
No Standard Model particle appears to be suitable to fulfil the role of dark matter, the non-luminous, weakly-interacting mass that appears to pervade our universe.
Similarly, the theory also provides no explanation dark energy, the force apparently driving the accelerating expansion of the universe.
Furthermore, our current understanding of physics provides no explanation for the observed predominance of matter over antimatter in our universe~\cite{matterAntimatterAMS}, given that the two should have been created in equal amounts during the Big Bang.
All this suggests that some larger theory, which we are yet to fully discover, underlies the Standard Model.

%%% How neutrinos can help us deal with this 
Observations of neutrinos may well provide an insight into so-called Beyond the Standard Model (BSM) physics.
Namely, the observation of neutrino oscillations~\cite{SNO, superK} and by extension neutrino mass was not predicted by the Standard Model.
As a result, careful measurement of neutrino properties is well placed to provide an insight into BSM physics.

%%% Experimental progress
In the 21\textsuperscript{th} century, great progress has been made in measuring some of the parameters controlling neutrino oscillations.
However, key some values remain ambiguous.
In order to determine these values a new generation of experiments is required.

The Deep Underground Neutrino Experiment (DUNE) is one of these `next-generation' experiments hoping to answer fundamental questions about neutrino physics.
In order to reach its full potential it is important that DUNE's measurement capabilities be assessed, so that its findings can be considered accurate.
This necessitates an effective understanding of DUNE's limitations, both of the detector itself and of the models DUNE uses of how neutrinos will interact with the detector.
This is the primary subject of this thesis.

%%% Explanations of what is in each chapter
\citechap{sec:theory} provides the theoretical background for much of this thesis.
We begin with a brief history of discoveries in the neutrino sector to provide
In particular, a theoretical overview of the neutrino, its interactions and the phenomenon of neutrino oscillations is given.
The chapter concludes by presenting the current experimental status of neutrino oscillations and prospects for the future.

\citechap{sec:dune} gives an overview of the DUNE experiment. 
This includes a description of the physics goals of the experiment and the technologies used in the detectors. 
A brief summary, including key technical details, is provided for each part of the experiment.

\citechap{sec:dune_lbl} expands on the previous chapter by providing an assessment of the systematic uncertainties assumed by DUNE.
Also explored is the effect these uncertainties have on the sensitivity of DUNE to various physical parameters involved in neutrino oscillations.

\citechap{sec:dune_ndrwt} provides an insight into hpw the gaseous argon component of the DUNE near detector can be used to correct for lack of knowledge about how neutrinos interact with matter.
The analysis also provides a demonstration of how this information can be used to help DUNE meet its physics goals.

\citechap{sec:hptpc_dtof_characterisation} describes the construction and characterisation of a time of flight system. 

\citechap{sec:hptpc_beam_flux} gives details of an analysis involving this time of flight system which took place at CERN in 2018.
This analysis demonstrates a novel method of producing a low energy proton beam.
This is of relevance when making some measurements relating to neutrino interactions.

\citechap{sec:protodune} provides a brief overview of the ProtoDUNE-SP detector, a prototype detector located at CERN.
Within the chapter, details of each of the major components of the detector are given.

\citechap{sec:pdune_calibration} provides a method of measuring the electron lifetime within the ProtoDUNE-SP detector.
Measurements of this electron lifetime are then compared with those from another source to give an estimate of a possible calibration error.
This calibration error is then propagated to DUNE far detector simulations where its effect on the measurement of neutrino oscillation parameters is assessed.

Finally, \citechap{sec:conclusions} presents general conclusions drawn from the work in this document and provides an outlook for future work.