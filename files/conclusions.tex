\chapter{Conclusions}
\label{sec:conclusions}

%%% General preamble
The work in this thesis shows that DUNE has the potential to make ground-breaking discoveries in the field of neutrino oscillations, details of which were laid out in \citechap{sec:theory}. 
Additionally, the current challenges in the science of neutrino oscillations were also stated.

\citechap{sec:dune} gave a detailed description of the DUNE experiment, discussing the detector technologies used throughout.
It also showed how these technologies make DUNE well-placed to search for phenomena such as CP violation in the lepton sector.
\citechap{sec:dune_lbl} expanded on the theme further by first giving a explanation of the systematic uncertainties used to calculate the sensitivities of DUNE. 
This included explanations of the how uncertainties concerning the neutrino flux, neutrino interaction model and detector model were estimated and implemented.
DUNE's sensitivities to the neutrino mass hierarchy and CP violation in the lepton sector were then presented, showing clearly that DUNE will quickly be able to resolve the neutrino mass hierarchy with no degeneracies.
Additionally, it was shown that after an exposure of \SI{336}{\kilo\tonne\mega\watt\year}, DUNE will approach discovery potential for CP violation if the value of the CP-violating phase is close to the maximal values.

Having presented these sensitivities in a situation where the neutrino interaction model is well known, \citechap{sec:dune_ndrwt} explored a scenario at DUNE where the observed neutrino interaction model is different than the expectation.
Here, the gaseous argon component of the DUNE near detector was used to correct for these interaction model deficiencies. 
It was shown that the low tracking threshold of this detector allows significant biases in some of the measured neutrino oscillation parameters to be corrected. 

\citechap{sec:hptpc_dtof_characterisation} described the characterisation of a time of flight system.
This time of flight system was then used in the analysis detailed in \citechap{sec:hptpc_beam_flux}.
Here, a novel method was used to produce a low energy (kinetic energy $<\SI{50}{\MeV}$) flux of protons from a higher energy beam.
This beam was impinged upon a high pressure gas time projection chamber with the intent of making measurements of the proton-argon cross-section.
If made, these sort of measurements would be highly pertinent to models of neutrino interactions with nuclei.
The analysis showed that the multiplicity of low energy protons passing through the active region of the detector was successfully increased from negligible levels to $4.9 \pm 0.1$ per spill.

It is understood that in order to understand DUNE's capabilities and limitations, some degree of prototyping is required.
One of the prototype detectors for the DUNE far detector (ProtoDUNE-SP) is described in \citechap{sec:protodune}.
\citechap{sec:pdune_calibration} presented an attempt to measure the electron lifetime in ProtoDUNE-SP using cosmic rays impinging on the detector.
These measurements were then combined with those derived from dedicated liquid argon purity monitors to determine the scale of possible biases in the electron lifetime and the particle energy scale.
This bias is propagated to far detector simulation in order to determine its effect on DUNE's sensitivities.
It was observed that a bias at the level observed in ProtoDUNE-SP data would have a negligible effect on DUNE's observation of CP violation.
However, if this bias were to be greater than expected it is observed that the resolution on the CP violating phase could be increased by 14\%.

%%% Outlook for DUNE and neutrino physics in general
The studies presented in this thesis show that in order for DUNE to achieve its physics goals it must be robust in many areas.
The work here also shows that particular attention must be paid to controlling the detector systematics.
Additionally, it is crucial that DUNE has the ability to accurately measure neutrino-nucleus interactions and that the models in this area continue to evolve with the project and the addition of new experimental inputs.

Overall, it is hoped that DUNE will be able to answer some of the fundamental questions surrounding neutrinos by the end of the decade and will be able to continue a vibrant physics program beyond this point.
