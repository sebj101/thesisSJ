\chapter{Historical context of neutrino physics}
\label{ch:history}

% Pauli's hypothesis of the neutrino

The existence of the neutrino as a particle carrying no electric charge was first hypothesised in 1930 by Wolfgang Pauli as a solution to issues observed in the spectrum produced by nuclear $\beta$-decay.
At the time, the existence of only 3 year subatomic particles was known, the electon, the proton and the photon~\cite{ideaOfNeutrino}.

In contrast to $\alpha$- and $\gamma$-decay, the energy spectrum of the emitted particle in $\beta$-decay is observed to be continuous~\cite{ideaOfNeutrino}.
Given that $\beta$-decay appeared to be a two-body decay (as $\alpha$- and $\gamma$-decay are) this seemed to violate the principle of conservation of energy.

To rectify this, Pauli proposed the existence of a neutral particle which escaped the nucleus without the detection during $\beta$-decay.
He originally named this particle the ``neutron'', although this name would be used for the neutral baryon after its discovery by Chadwick in 1932~\cite{neutronDiscovery} and the neutrino would be called its modern name.

We now know that an electron neutrino (antineutrino) is emitted during $\beta^{+}$- ($\beta^{-}$-) decay where the generic equation for the $\beta^{-}$ decay of an atom is
\begin{equation}
  ^{A}_{Z}X~\rightarrow~^{A}_{Z+1}X' + e^{-} + \bar{\nu}_{e} \,,
\end{equation}
where $A$ is the mass number of the atom, and $Z$ is the atomic number. 

% Experimental discovery by Cowan and Reines
Experimental confirmation of the neutrino's existence came in 1956 where the reaction
\begin{equation}
  \bar{\nu}_{e} + p \rightarrow e^{+} + n
  \label{eq:cowanReines}
\end{equation}
was observed, utilising the large neutrino flux from a nuclear reactor incident on the protons in a tank of water sandwiched between scintillation counters.
This water contained a solution of cadmium chloride which emits a photon upon neutron capture.
In the event of the reaction shown in eq.~(\ref{eq:cowanReines}), the outgoing positron would annihilate with an electron in the water, resulting in a prompt coincidence signal.
An additional check was made by requiring a delayed photon from the capture of the outgoing neutron~\cite{cowanReines}.

% Discovery of numu
Up to this point, only electron neutrinos had been observed (counterparts to the electron).
However, in 1962 the existence of a neutrino corresponding to the muon was confirmed at Brookhaven Alternatng Gradient Synchrotron.
These muon neutrinos and antineutrinos ($\nu_{\mu}$ and $\bar{\nu}_{\mu}$) were produced by colliding a beam of protons with a beryllium target, producing charged pions and kaons.
These mesons subsequently decayed via the channels
\begin{align}
    K^{\pm} \rightarrow \mu^{\pm} + \nu_{\mu}(\bar{\nu}_{\mu}) \\
    \pi^{\pm} \rightarrow \mu^{\pm} + \nu_{\mu}(\bar{\nu}_{\mu})
\end{align}
to produce the desired neutrinos and antineutrinos.

The resulting neutrino interactions were then observed in a spark chamber.
Interactions arising from muon neutrinos and antineutrinos were identified by the prescence of one or more long tracks originating from within the fiducial volume of the detector at the same time as the beam pulse~\cite{numuDiscovery}.   

% nutau discovery from LEP measurements and eventual confirmation from DONUT

% C. S. Wu's discovery of neutrino helicity and parity violation?

% Solar neutrino problem and discovery of oscillations

% Current state of PMNS matrix


