% I may change the way this is done in a future version, 
%  but given that some people needed it, if you need a different degree title 
%  (e.g. Master of Science, Master in Science, Master of Arts, etc)
%  uncomment the following 3 lines and set as appropriate (this *has* to be before \maketitle)
% \makeatletter
% \renewcommand {\@degree@string} {Master of Things}
% \makeatother

\title{Using charged particle test beams to constrain systematic uncertainties for the DUNE experiment}
\author{Sebastian Jones}
\department{Department of Physics \& Astronomy}

\maketitle
\makedeclaration

\begin{abstract} % 300 word limit

The existence of non-zero neutrino masses is the one unexplained feature of the Standard Model.
As such, neutrinos warrant further investigation.

The Deep Underground Neutrino Experiment (DUNE) is a long-baseline neutrino oscillation experiment currently in its construction phase.
Upon completion it aims to resolve several questions regarding the properties of neutrinos by making precise measurements of muon neutrino disappearance and electron neutrino appearance at a baseline of \SI{1300}{\km}.
Specifically, DUNE will aim to find evidence of Charge-Parity symmetry (CP) violation in the lepton sector and determine the ordering of the neutrino mass eigenstates (the neutrino mass hierarchy).

In order to accomplish these goals, DUNE's systematic uncertainties must be understood and it is this that is the primary subject of this thesis.
Initially, studies are presented showing the effect of DUNE's nominal systematic uncertainties on its sensitivity to CP violation and the neutrino mass hierarchy, where the importance of controlling the detector systematics is highlighted. 

Following this, methods of controlling systematics resulting from neutrino cross-sections are presented, both utilising a high pressure gas time projection chamber.
This includes a novel method for producing a low energy proton beam for use in neutrino experiments.

Finally, a measurement of the electron lifetime in the ProtoDUNE-SP detector is presented.
This lifetime is compared to the values obtained from dedicated purity monitors and translated to an energy scale uncertainty.
The effect of this uncertainty on DUNE's measurement of the neutrino oscillation parameters is assessed.

\end{abstract}

\begin{impactstatement}

The work presented in this thesis was performed for two collaborations: the DUNE collaboration and the HPTPC collaboration.

Within the DUNE collaboration the studies in this thesis will hopefully continue to inform the design choices of the experiment (in particular the DUNE near detector) as it develops. 
Additionally, the understanding of the systematic uncertainties used in any future DUNE long-baseline analysis will be furthered by the work contained within this thesis.

Looking at the work performed for the HPTPC collaboration, a novel method is presented for generating a low energy proton beam.
This method has great utility for the neutrino physics community as a whole as such beams can be used to make valuable measurements of proton-nucleus cross-sections which are used to inform models of neutrino interaction.
Given that proton beams in this energy range are in short supply, a proven method of generating them from an existing beam could have great benefit to the neutrino physics community.

In the broader sense, the contribution of DUNE to the world primarily concerns answering the most fundamental of physics questions.
Determining the properties of neutrinos will hopefully deepen our understanding of the physical world and the laws that underpin it.

\end{impactstatement}

\begin{acknowledgements}
Acknowledge all the things!
\end{acknowledgements}

\setcounter{tocdepth}{2} 
% Setting this higher means you get contents entries for
%  more minor section headers.

\tableofcontents
\listoffigures
\listoftables

