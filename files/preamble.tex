% I may change the way this is done in a future version, 
%  but given that some people needed it, if you need a different degree title 
%  (e.g. Master of Science, Master in Science, Master of Arts, etc)
%  uncomment the following 3 lines and set as appropriate (this *has* to be before \maketitle)
% \makeatletter
% \renewcommand {\@degree@string} {Master of Things}
% \makeatother

\title{Using charged particle test beams to constrain systematic uncertainties for the DUNE experiment}
\author{Sebastian Jones}
\department{Department of Physics \& Astronomy}

\maketitle
\makedeclaration

\begin{abstract} % 300 word limit

The Deep Underground Neutrino Experiment (DUNE) is a long-baseline neutrino oscillation experiment currently in its construction phase.
Upon completion it aims to resolve several questions regarding the properties of neutrinos.
Specifically, DUNE will aim to find evidence of Charge-Parity symmetry (CP) violation in the lepton sector and determine the ordering of the neutrino mass eigenstates (the neutrino mass hierarchy).

In order to accomplish these goals, DUNE's systematic uncertainties must be understood and it is this that is the primary subject of this thesis.
Initially, studies are presented showing the effect of DUNE's nominal systematic uncertainties on its sensitivity to CP violation and the neutrino mass hierarchy, where the importance of controlling the detector systematics is highlighted. 

Following this, methods of controlling systematics resulting from neutrino cross-sections are presented, both utilising a high pressure gas time projection chamber.
This includes a novel method for producing a low energy proton beam for use in neutrino experiments.

Finally, a measurement of the electron lifetime in the ProtoDUNE-SP detector is presented.
This lifetime is compared to the values obtained from dedicated purity monitors and translated to an energy scale uncertainty.
The effect of this uncertainty on DUNE's measurement of the neutrino oscillation parameters is assessed.



\end{abstract}

\begin{impactstatement}

	UCL theses now have to include an impact statement. \textit{(I think for REF reasons?)} The following text is the description from the guide linked from the formatting and submission website of what that involves. (Link to the guide: {\scriptsize \url{http://www.grad.ucl.ac.uk/essinfo/docs/Impact-Statement-Guidance-Notes-for-Research-Students-and-Supervisors.pdf}})

\begin{quote}
The statement should describe, in no more than 500 words, how the expertise, knowledge, analysis,
discovery or insight presented in your thesis could be put to a beneficial use. Consider benefits both
inside and outside academia and the ways in which these benefits could be brought about.

The benefits inside academia could be to the discipline and future scholarship, research methods or
methodology, the curriculum; they might be within your research area and potentially within other
research areas.

The benefits outside academia could occur to commercial activity, social enterprise, professional
practice, clinical use, public health, public policy design, public service delivery, laws, public
discourse, culture, the quality of the environment or quality of life.

The impact could occur locally, regionally, nationally or internationally, to individuals, communities or
organisations and could be immediate or occur incrementally, in the context of a broader field of
research, over many years, decades or longer.

Impact could be brought about through disseminating outputs (either in scholarly journals or
elsewhere such as specialist or mainstream media), education, public engagement, translational
research, commercial and social enterprise activity, engaging with public policy makers and public
service delivery practitioners, influencing ministers, collaborating with academics and non-academics
etc.

Further information including a searchable list of hundreds of examples of UCL impact outside of
academia please see \url{https://www.ucl.ac.uk/impact/}. For thousands more examples, please see
\url{http://results.ref.ac.uk/Results/SelectUoa}.
\end{quote}
\end{impactstatement}

\begin{acknowledgements}
Acknowledge all the things!
\end{acknowledgements}

\setcounter{tocdepth}{2} 
% Setting this higher means you get contents entries for
%  more minor section headers.

\tableofcontents
\listoffigures
\listoftables

