% I may change the way this is done in a future version, 
%  but given that some people needed it, if you need a different degree title 
%  (e.g. Master of Science, Master in Science, Master of Arts, etc)
%  uncomment the following 3 lines and set as appropriate (this *has* to be before \maketitle)
% \makeatletter
% \renewcommand {\@degree@string} {Master of Things}
% \makeatother

\title{Using charged particle test beams to constrain systematic uncertainties for the DUNE experiment}
\author{Sebastian Jones}
\department{Department of Physics \& Astronomy}

\maketitle
\makedeclaration

\begin{abstract} % 300 word limit

The existence of non-zero neutrino masses is the one unexplained feature of the Standard Model.
As such, neutrinos warrant further investigation.

The Deep Underground Neutrino Experiment (DUNE) is a long-baseline neutrino oscillation experiment currently in its construction phase.
Upon completion it aims to resolve several questions regarding the properties of neutrinos by making precise measurements of muon neutrino disappearance and electron neutrino appearance at a baseline of \SI{1300}{\km}.
Specifically, DUNE will aim to find evidence of Charge-Parity symmetry (CP) violation in the lepton sector and determine the ordering of the neutrino mass eigenstates (the neutrino mass hierarchy).

In order to accomplish these goals, DUNE's systematic uncertainties must be understood and it is this that is the primary subject of this thesis.
Initially, studies are presented showing the effect of DUNE's nominal systematic uncertainties on its sensitivity to CP violation and the neutrino mass hierarchy, where the importance of controlling the detector systematics is highlighted. 

Following this, methods of controlling systematics resulting from neutrino cross-sections are presented, both utilising a high pressure gas time projection chamber.
This includes a novel method for producing a low energy proton beam for use in neutrino experiments.

Finally, a measurement of the electron lifetime in the ProtoDUNE-SP detector is presented.
This lifetime is compared to the values obtained from dedicated purity monitors and translated to an energy scale uncertainty.
The effect of this uncertainty on DUNE's measurement of the neutrino oscillation parameters is assessed.

\end{abstract}

\begin{impactstatement}

The work presented in this thesis was performed for two collaborations: the DUNE collaboration and the HPTPC collaboration.

Within the DUNE collaboration the studies in this thesis will hopefully continue to inform the design choices of the experiment (in particular the DUNE near detector) as it develops. 
Additionally, the understanding of the systematic uncertainties used in any future DUNE long-baseline analysis will be furthered by the work contained within this thesis.

Looking at the work performed for the HPTPC collaboration, a novel method is presented for generating a low energy proton beam.
This method has great utility for the neutrino physics community as a whole as such beams can be used to make valuable measurements of proton-nucleus cross-sections which are used to inform models of neutrino interaction.
Given that proton beams in this energy range are in short supply, a proven method of generating them from an existing beam could have great benefit to the neutrino physics community.

In the broader sense, the contribution of DUNE to the world primarily concerns answering the most fundamental of physics questions.
Determining the properties of neutrinos will hopefully deepen our understanding of the physical world and the laws that underpin it.

\end{impactstatement}

\begin{acknowledgements}
	
I would like to begin by thanking my supervisor, Ryan Nichol, for his supervision over the course of this PhD. 
Without his considerable neutrino physics expertise to aid me (as well as his patience in answering my questions) this work would not have been possible.

Science is a collaborative endeavour at its heart and therefore I would like to thank the various members of the DUNE and HPTPC collaborations who have played a part, large or small, in the course of this work.
My special thanks goes to two people: Chris Marshall, who provided both the inspiration and much of the guidance for the study in \citechap{sec:dune_ndrwt}, and Toby Nonnenmacher, my principal co-collaborator for the study featured in \citechap{sec:hptpc_beam_flux}.

Going further back I would like to acknowledge the contributions of the supervisors of my master's project, Stefan S\"oldner-Rembold and Andrzej Szelc.
Together, they encouraged me to believe that I had the capacity to pursue a PhD in high energy physics.

On a personal level I must thank my friends (both in and out of physics) and family for their support in this endeavour.
In particular, my girlfriend Georgia deserves a special mention for the motivation she has provided during this PhD, as well as the light she has brought into my life when it has seemed as though the work is getting too much.

Finally, I would like to express my gratitude to STFC for supplying the studentship that has made this PhD possible.

\end{acknowledgements}

\setcounter{tocdepth}{2} 
% Setting this higher means you get contents entries for
%  more minor section headers.

\tableofcontents
\listoffigures
\listoftables

