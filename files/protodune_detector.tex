\chapter{The ProtoDUNE detector}
\label{sec:protodune}

The construction of the DUNE far detector represents a significant increase in scale for liquid argon time projection chambers (LAr TPCs).
Prior to the construction of ProtoDUNE Single Phase (ProtoDUNE-SP), the largest constructed LAr TPC was the ICARUS T600 detector~\cite{icarus}, originally operated at the INFN Gran Sasso Laboratory (LNGS).
This detector contained roughly \SI{600}{\tonne} of liquid argon in a single cryostat.
In comparison, each DUNE far detector module is expected to contain roughly \SI{17}{\kilo\tonne} of liquid argon.
 
Additionally, the proposed lifetime of DUNE (over 15 years) and its relative inaccessibility (located as it is \textbf{SOME NUMBER OF METRES} underground) means that the detector technologies must be shown to have long term stability.
 
It is therefore necessary to complete extensive prototyping of the technologies used in the DUNE far detector


\section{Motivation}
\label{sec:protodune:motivation}

\section{Time Projection Chamber}
\label{sec:protodune:tpc}

\section{Photon detection system}
\label{sec:protodune:pds}

\section{Beamline}
\label{sec:protodune:beamline}