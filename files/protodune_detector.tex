\chapter{The ProtoDUNE Single Phase detector}
\label{sec:protodune}

Over the course of 2018 two detectors were assembled at CERN to act as prototypes for the the DUNE far detector modules.
One of these prototypes was a single-phase LAr TPC while the other was a dual-phase LAr TPC.
The single-phase prototype is termed ProtoDUNE Single Phase (ProtoDUNE-SP) and will be discussed in this chapter.
Extensive prototyping of the technologies used in the DUNE far detector prior to the construction of far detector modules is necessary for a variety of reasons.

The construction of the DUNE far detector represents a significant increase in scale for liquid argon time projection chambers (LAr TPCs).
Prior to the construction of ProtoDUNE-SP, the largest constructed monolithic LAr TPC was the ICARUS T600 detector~\cite{icarus}, originally operated at the INFN Gran Sasso Laboratory (LNGS).
This detector contained roughly \SI{600}{\tonne} of liquid argon in a single cryostat.
In comparison, each DUNE far detector module is expected to contain roughly \SI{17}{\kilo\tonne} of liquid argon.

The DUNE far detector modules will also also have significantly larger maximum drift distances than previously constructed LAr TPCs.
The maximum drift distance in a DUNE single-phase far detector module will be \SI{3.6}{\m} compared with a maximum drift distance in ICARUS T600 of \SI{1.5}{\m}.
This increased drift length necessitates a higher argon purity to achieve an electron lifetime that will enable ionisation charge from throughout the detector to be seen.

Additionally, the proposed lifetime of DUNE (over 15 years) and its relative inaccessibility (located as it is \SI{1.5}{\kilo\metre} underground) means that the detector technologies used must be shown to have long term stability.

This prototype uses many of the components expected to be used in a DUNE FD module such as the Anode Plane Assemblies (see \citesec{sec:dune:fd:modules:singlephase}) expected to be used in the FD single-phase modules.
Furthermore, the drift distance in ProtoDUNE-SP is the same as the expected drift distance in a DUNE single-phase FD module.

% Basic information
ProtoDUNE-SP consists of a membrane cryostat with a cuboidal interior shape.
This cuboid measures \SI{7.9}{\m} in height and \SI{8.5}{\m} in the other two dimensions.
Within this cryostat, there are two TPCs measuring \SI{6.1}{\m} in height, \SI{3.6}{\m} in the drift direction and \SI{7.1}{\m} in the remaining direction.
These two TPCs are separated by a central cathode plane.
Together these TPCs contain an instrumented mass of \SI{419}{\tonne} of liquid argon.
Further details of the time projection chambers are given in \citesec{sec:protodune:tpc}.

In order to make the larger drift distances planned for the DUNE FD modules feasible, a long electron lifetime (and therefore a high liquid argon purity) is required.
Additionally, in order for calorimetric measurements of charged particles to be made, the electron lifetime must be accurately known.
To this end, three liquid argon purity monitors are installed in the ProtoDUNE-SP cryostat.
The operating principles of these purity monitors are described in \citesec{sec:protodune:prms}.

The detector is exposed to charged particles in the momentum range \SIrange{0.3}{7}{\GeV\per\clight}~\cite{protodunePerformance}.
This will allow measurements of the detector response to these particles to be made.
These measurements will inform future DUNE physics analyses.
Furthermore, it will be possible to make cross-section measurements for these particles on liquid argon.
These particles are derived from a tertiary beam which then passes through the H4-VLE beam line where the particles are momentum selected.
This beam line contains a variety of instrumentation with which the particle species is determined.
More details of the beam and beam line are given in \citesec{sec:protodune:beam}.

A diagram of the detector with several key components labelled is shown in \citefig{fig:pdspDiag}.
Additionally, the detector is instrumented with a photon detection system in order to measure scintillation light from charged particle interactions. 
This light provides timing information and also allows for an independent measurement of particle energy.
Further details are given in \citesec{sec:protodune:pds}.

\begin{figure}[h]
	\centering
	\includegraphics[width=.6\linewidth]{files/figures/protodune_detector/pdspDiag}
	\caption[Captioned diagram of the ProtoDUNE-SP cryostat interior]{Diagram of the ProtoDUNE-SP cryostat interior from~\cite{protodunePerformance}. Various key objects are labelled.}
	\label{fig:pdspDiag}
\end{figure}

Finally, two of the exterior cryostat faces are covered by a cosmic ray tagger.
This cosmic ray tagger measures the time and position of a sample of the significant flux of cosmic rays striking the detector (approximately \SI{20}{\kilo\hertz}~\cite{protodunePerformance}), allowing independent measurements of TPC performance to be made.
Further details are given in \citesec{sec:protodune:crt}.

\section{Time Projection Chamber}
\label{sec:protodune:tpc}

As detailed above and in \citesec{sec:dune:fd:modules:singlephase} each DUNE TPC is formed from a cathode and an anode, across which a voltage is applied.
Similar to the plan for the DUNE FD single-phase modules, the cathode and anode are themselves made up of smaller units, the CPAs and APAs respectively.

Each anode plane is formed from three APAs each of which consists of a steel frame to which 4 wire planes are attached.
These wire planes are each separated from the following plane by a distance of \SI{4.79}{\mm}.
The wrapping scheme, which is the same as that planned for the DUNE FD modules, is shown in \citefig{fig:wireWrapping}.
From a drifting electron's point of view the first plane encountered is the ground (G) plane followed by the two induction planes (U and then V) which are all passed without the electron being captured.
Finally, the drift electron reaches the collection plane (X) where it is collected.
On either side of the APA, a bronze mesh is attached to act as a grounded shield plane for the wires.

Simulated waveforms from DUNE for each of the wire planes are shown in \citefig{fig:wireSignals}. 
The waveforms will have the same features as those in ProtoDUNE-SP.
For the induction planes a bipolar signal is produced as the electrons initially move towards the wire before passing it.
The collection plane produces a unipolar signal as the electrons are collected.

\begin{figure}[h]
	\centering
	\includegraphics[width=.6\linewidth]{files/figures/protodune_detector/wireSignals}
	\caption[Simulated waveforms in the DUNE detector for MIPs travelling parallel to the wire plane]{Simulated waveforms in the DUNE detector for MIPs travelling parallel to the wire plane, from~\cite{tdrVol2}. Here, as in ProtoDUNE-SP, the U and V wires are the induction planes and the X wire is the collection plane.}
	\label{fig:wireSignals}
\end{figure}

Bias voltages are applied to the wire planes to ensure that ionisation electrons drift past the induction planes and are collected on the collection planes.
The voltages used are as follows: $V_{G} = \SI{-665}{\volt}$, $V_{U} = \SI{-370}{\volt}$, $V_{V} = \SI{0}{\volt}$ and $V_{X} = \SI{+820}{\volt}$~\cite{protodunePerformance}.

The cathode plane is formed from six CPAs, each \SI{6.1}{\m} in height and \SI{1.15}{\m} wide.
These CPAs are constructed from highly resistive materials which ensures that they do not rapidly discharge in the event of an electrical breakdown.
Such a discharge could damage the TPC electronics.

The cathode plane is biased at a voltage of \SI{-180}{\kilo\volt} which generates a field of \SI{500}{\volt\per\cm} across each drift volume.
This field is gradually stepped down in 60 steps by the field cage, thus ensuring that the field remains uniform between the anode and cathode plane.

This drift field results in a nominal electron drift velocity of \SI{1.59}{\mm\per\micro\second} which translates to a maximum drift time of \SI{2.25}{\micro\second}.

\section{Purity monitors}
\label{sec:protodune:prms}

ProtoDUNE-SP contains three purity monitors designed to provide independent measurements of the liquid argon purity.
These purity monitors were the same design as the ones operated in the ICARUS T600 detector~\cite{icarus}.
They are located exterior to the TPC at varying heights in the cryostat.

\begin{figure}[h]
	\centering
	\includegraphics[width=.6\linewidth]{files/figures/protodune_detector/prmDiag}
	\caption[Circuit diagram of a typical purity monitor used in a liquid argon experiment]{Circuit diagram of a liquid argon purity monitor, from~\cite{Adamowski_2014}.}
	\label{fig:prmDiag}
\end{figure} 

A circuit diagram of a typical purity monitor used in liquid argon experiments is shown in \citefig{fig:prmDiag}.
The general operating procedure is as follows: a light pulser (in the case of the monitors used in ProtoDUNE-SP, a xenon lamp) briefly illuminates the photo-cathode, liberating electrons via the photo-electric effect.
A mesh grid (labelled ground grid in \citefig{fig:prmDiag} but also referred to as the cathode grid) is held at ground potential.
The liberated electrons drift towards the cathode grid, inducing a signal on the cathode which is then read out.
A uniform electric field, maintained by a series of field rings, then drifts these electrons towards the anode.
Along the way, some of these drift electrons are absorbed by impurities in the argon.
Upon crossing another mesh grid at the end of the drift volume (the anode grid), the electrons induce a signal on the anode.

The anode and cathode signals are then converted into charges, $Q_{A}$ and $Q_{C}$ respectively.
The ratio of these two charges can then be used to measure the electron lifetime via the equation
\begin{equation}
\frac{Q_{A}}{Q_{C}} = e^{-t/\tau}
\end{equation}
where $t$ is the drift time, which can be determined from the electric field and $\tau$ is the electron lifetime.

\section{Beam line}
\label{sec:protodune:beam}

The beam which is used in the ProtoDUNE-SP beam tests is originally derived from bunches of \SI{400}{\GeV\per\clight} protons from the CERN Super Proton Synchrotron (SPS) which then strike a beryllium target producing a secondary beam of hadrons with a momentum of \SI{80}{\GeV\per\clight}.
The secondary beam then impinges on a further target generating a tertiary, very low energy (VLE) beam with a momentum range of \SIrange{0.3}{7}{\GeV\per\clight}~\cite{protodunePerformance}.

This beam then enters the H4-VLE beam line~\cite{h4vle} which momentum selects the particles and transports them to the detector.
This beam line also contains instrumentation which allows for the momentum analysis and particle identification of the particles in the beam which is covered in further detail in Ref.~\cite{boothBeamLineInstr}.
A schematic diagram of both the beam line instrumentation used in the H4 VLE beam line is shown in \citefig{fig:h4vleDiag}.

\begin{figure}[h]
	\centering
	\includegraphics[width=\linewidth]{files/figures/protodune_detector/h4vleDiag}
	\caption[Schematic diagram of the beam line instrumentation used in the H4-VLE beam line]{Schematic diagram of the beam line instrumentation used in the H4-VLE beam line, from~\cite{protodunePerformance}. Shown are trigger counters (labelled XBTF), beam profile monitors (labelled XBPF), Cherenkov detectors (labelled XCET) and bending magnets (green triangles).}
	\label{fig:h4vleDiag}
\end{figure}

In order to identify the species of particles both time of flight measurements and Cherenkov detectors are used.
This time of flight measurement is made between the upstream and downstream trigger counters, a baseline of \SI{28.575}{\m}.
The precision of this system is measured as around \SI{900}{\pico\second}~\cite{boothBeamLineInstr}.

The momentum of particles is determined using the beam profile monitors.
Together with the bending magnets, these can be used as a momentum spectrometer by measuring the deflection of particles travelling through the beam line.

Additionally, two Cherenkov detectors are also used for particle type reconstruction.
This is useful at momenta where the particle species cannot be determined due to the similar masses of the particles in question.
These detectors consist of two roughly \SI{2}{\m} long $\text{CO}_{2}$ gas-filled volumes. Further details are available in Ref.~\cite{vleCherenkov}.
The gas in the upstream detector is held at a pressure of \SI{15}{\bar} while that in the downstream detector is held at a pressure of \SI{5}{\bar}.

These two pressures ensure that the momentum threshold for the production of Cherenkov light is different between the two detectors and thus, at a given momentum, different particle species will produce Cherenkov light in each detector.
\citefig{fig:cherenkovThresh} shows the threshold pressures required to produce Cherenkov light as a function of momentum and particle type.
One can see that at momenta of more than approximately \SI{1}{\GeV} some particles will trigger at least one Cherenkov detector.

\begin{figure}[h]
	\centering
	\includegraphics[width=.6\linewidth]{files/figures/protodune_detector/cherenkovThresh}
	\caption[Cherenkov gas pressure thresholds as a function of particle momentum and species]{Cherenkov gas pressure thresholds as a function of particle momentum and species, from~\cite{boothBeamLineInstr}. The dashed red lines indicate the operating pressures of the two Cherenkov detectors in the H4-VLE beam line.}
	\label{fig:cherenkovThresh}
\end{figure}

The trigger logic for particle identification involves a combination of time of flight cuts and detection in the Cherenkov detectors depending on the selected beam momentum.

Upon reaching the detector, a `beam plug' ensures that the particles enter the TPC while passing through minimal amounts of inactive material.
This inactive material would otherwise include not only the cryostat wall but also \SI{40}{\cm} of inactive liquid argon exterior to the TPC.

This beam plug is located about half way up the detector and approximately \SI{30}{\cm} off-centre in the drift direction.
It consists of a number of alternating stainless steel and fibreglass rings which together form a sealed cylinder with the ends capped with fibreglass plates.
The cylinder extends from the cryostat until about \SI{5}{\cm} inside the field cage boundary.
This cylinder is filled with nitrogen at a pressure of \SI{1.3}{\bar} which counters the pressure of the liquid argon without significantly increasing the probability of particle interaction~\cite{protodunePerformance}.

The beam composition varies as a function of momentum and is a mixture of protons, muons, pions, kaons and electrons.
\citefig{fig:beamComp} shows the measured beam composition for two different nominal beam momenta.
One can see that the kaon content is negligible at \SI{1}{\GeV\per\clight} due to decay in flight while at higher momenta it becomes hard for the beam line instrumentation to distinguish between electrons, muons and pions using time of flight information.

\begin{figure}[h]
	\begin{minipage}[t]{.49\linewidth}
		\includegraphics[width=\linewidth]{files/figures/protodune_detector/beamComp1GeV}
	\end{minipage}
	\hfill
	\begin{minipage}[t]{.49\linewidth}
		\includegraphics[width=\linewidth]{files/figures/protodune_detector/beamComp6GeV}
	\end{minipage}
	\caption[Time of flight spectra as measured in the H4-VLE beam line, separated using particle identification]{Time of flight spectra as measured in the H4-VLE beam line, from~\cite{protodunePerformance}. The spectra are separated using particle identification techniques such as time of flight measurements and the Cherenkov detectors. Left: nominal beam momentum of \SI{1}{\GeV\per\clight}. Right: nominal beam momentum of \SI{5}{\GeV\per\clight}.}
	\label{fig:beamComp}
\end{figure}

\section{Photon detection system}
\label{sec:protodune:pds}

Light is collected by photosensors within the TPC for two primary reasons. 
One is in order to give timing information for charged particle interactions.
This timing information can be used to determine an interaction time and drift position, allowing corrections for position dependent factors.
The other is in order to gain independent measurements of particle energy from the amount of emitted light.

To this end, ProtoDUNE-SP contains 60 optical modules that together make up the photon detection system (PDS).
These modules have dimensions of \SI{8.6 x 220 x 0.6}{\cm}.
Each of these modules is integrated into an APA (locating the modules exterior to the field cage would block too much light) with the spacing between modules kept equal.

Within these space constraints there are three different designs of optical module~\cite{protoduneSP_tdr}.
However, they all function on the same principle. 
The argon scintillation light, which has a characteristic wavelength of \SI{128}{\nano\metre} is incident on some wavelength shifting material in the module.
This has the effect of shifting the photon wavelengths to a region where they can be more easily detected by SiPMs. 

\section{Cosmic ray tagger}
\label{sec:protodune:crt}

The upstream and downstream faces of the cryostat are mostly covered by a number of modules making up the cosmic ray tagger (CRT).
Therefore, cosmic rays that strike both an upstream and a downstream CRT module will be travelling roughly parallel to the TPC readout planes.
These tagged cosmic rays will have a known time, position and direction.

A schematic drawing of a number of CRT modules arranged into a larger panel is shown in \citefig{fig:crtModules}.
These modules were originally part of the outer veto of the Double Chooz experiment~\cite{dcIV}.
Each module consists of 64 scintillator strips, each \SI{5}{\cm} wide and \SI{365}{\cm} long.
These strips are arranged parallel to one another.
Light produced by the scintillator strips is read out by Hamamatsu M64 multi-anode PMTs.

\begin{figure}[h]
	\centering
	\includegraphics[width=.6\linewidth]{files/figures/protodune_detector/crtModules}
	\caption[Drawing of several CRT modules arranged into a larger plane]{Drawing of several CRT modules arranged into a larger plane, from~\cite{protodunePerformance}. Dimensions of the modules are also shown.}
	\label{fig:crtModules}
\end{figure}

This arrangement gives spatial coordinates in one dimension only, so two sets of modules are overlaid on another two modules, rotated by \ang{90}. 
In this manner, particles detected in two overlapping modules will have definite three dimensional hit positions (the third coordinate comes from the fixed position of the module).

