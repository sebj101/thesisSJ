\chapter{Characterisation of the downstream time of flight system}
\label{ch:hptpc_dtof_characterisation}

The time taken by a particle to traverse a distance $L$ is given by
\begin{equation}
  t = L \sqrt{ \frac{m^{2}}{p^{2}} + \frac{1}{c^{2}} } \, ,
\end{equation}
where $m$ is the mass of the particle, $p$ is the particle momentum and $c$ is the speed of light in a vacuum.

\citesec{ch:hptpc_dtof_characterisation:dtof} describes the DToF while \citesec{ch:hptpc_dtof_characterisation:characterisation} describes the characterisation of the PMTs and bars.

\section{The downstream time of flight system (DToF)}
\label{ch:hptpc_dtof_characterisation:dtof}

The DToF consists of 10 bars made up of Nuvia NuDET plastic scintillator which has a wavelength of maximum emission of \SI{425}{\milli\metre} and a decay time constant of \SI{2.5}{\nano\second}~\cite{nuvia}.
Each one of these bars measures $\SI{10}{\centi\metre} \times \SI{1}{\centi\metre} \times \SI{140}{\centi\metre}$.
A diagram of one such bar is shown in \citefig{fig:barDiag}.
The curved cutouts at either end of the bar each accomodate a 5'' Hamamatsu Photonics R6594 PMT~\cite{hamamatsu}.

\begin{figure}
  \centering
  \includegraphics[width=.8\linewidth]{files/figures/hptpc_dtof_characterisation/barDiag}
  \caption[HPTPC DToF bar diagram]{Computer generated drawing of one of the scintillator bars from two views.}
  \label{fig:barDiag}
\end{figure}

The bars and PMTs are arranged in two rows of five bars to ensure complete coverage for any particles paimpinging on the detector.
This gives the DToF a total active area of $\SI{1.40}{\metre} \times \SI{0.78}{\metre}$.
\citefig{fig:dtofDiag} shows a diagram of the DToF system along with various dimensions of sections of the detector.
Each of the bars is wrapped individually in a reflective milar sheet in order to maximise the light yield as well as keeping external light out of the bar.

\begin{figure}[h]
  \centering
  \includegraphics[width=.5\linewidth]{files/figures/hptpc_dtof_characterisation/dstofFront}
  \caption[Diagram of the HPTPC DToF]{Diagram of the DToF along with measurements.}
  \label{fig:dtofDiag}
\end{figure}

Each of the PMTs is attached to the scintillator bar via 3D-printed support which is glued to the bar.
A layer of optical grease is applied to the end of the scintillator bar where it meets the PMT.
This ensures good optical coupling between the PMT and scintillator bar and minimises light loss.

The vertical position of a particle striking the DToF can be measured by observing in which bar a signal is observed.
The horizontal position of the same particle can be inferred by measuring the time difference between signals being observed in PMTs at opposite ends of the same bar.

However, in order to effectively convert this time measurement to one of position, two factors must be measured.
Firstly, the relationship between the time difference in signals being observed in PMTs and the position of a particle along the bar must be measured.
Secondly, the time (and therefore position) resolution must be measured.

\section{Characterisation of the bars}
\label{ch:hptpc_dtof_characterisation:characterisation}




