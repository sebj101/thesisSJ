\chapter{The Deep Underground Neutrino Experiment}
\label{ch:dune}

The Deep Underground Neutrino Experiment (DUNE) is a long-baseline neutrino oscillation experiment currently under construction at two sites in the USA~(\cite{tdrVol1}, \cite{tdrVol2}, \cite{tdrVol3}, \cite{tdrVol4}).
The near detector site which will host the beam production facilities and the multi-component near detector is located at Fermilab, Illinois while the far detector site will be located at Sanford Underground Research Facility (SURF), South Dakota.
This will provide a baseline of approximately 1300~km.
A cartoon showing the arrangement of the detectors is shown in Figure~\ref{fig:duneCartoon}.

\begin{figure}[h]
  \centering
  \includegraphics[width=.9\linewidth]{files/figures/dune_detector/duneCartoon}
  \caption{Cartoon showing the arrangement of DUNE (not to scale) with the neutrino beam propagating from Fermilab to SURF along its 1300~km baseline. Taken from~\cite{tdrVol1}.}
  \label{fig:duneCartoon}
\end{figure}

In this chapter, the DUNE detector will be discussed.
Section~\ref{sec:dune:overview} will provide an overview of the detector, its motivation and its physics goals.
Section~\ref{sec:dune:fd} will discuss the design of the far detector.
Section~\ref{sec:dune:nd} will discuss the design of the near detector and Section~\ref{sec:dune:beam} will discuss the design of the beam.

\section{Overview}
\label{sec:dune:overview}
As discussed in Section~\ref{sec:theory:currentState}, several questions remain in the field of neutrino oscillations.
Namely, the ordering of the neutrino mass eigenstates is not yet known and the value of $\delta_{CP}$ in the PMNS matrix remains unknown, which could imply that neutrinos violate Charge-Parity symmetry if it is proven that $\delta_{CP} \neq 0, \pi$.
Additionally, DUNE aims to be able to resolve the octant of $\theta_{23}$ (i.e. is $\theta_{23} > \pi/4$ or $< \pi/4$?) and usher in a new era of precision measurements of the PMNS matrix. 
DUNE aims to answer these questions, among others.

Due to the large volume of the far detector, it is also designed to observe neutrinos from a galactic core-collapse supernova if one should occur during the running time of the experiments.
This same large far detector mass will also allow it to search for certain Beyond the Standard Model (BSM) processes such as baryon number violation via proton decay.
A range of other BSM processes will also be explored by the experiment.

% Physics goals
% Oscillations first and foremost are the most important - refer to what is in the historical context chapter
DUNE aims to achieve its oscillation physics goals by simultaneously measuring the rates of $\nu_{\mu}$, $\bar{\nu}_{\mu}$, $\nu_{e}$ and $\bar{\nu}_{e}$ interactions in the near and far detectors as a function of neutrino energy.
Enhanced fluxes of $\bar{\nu}_{\mu}$ and $\bar{\nu}_{e}$ are achieved by changing the polarity of the focussing horns used to produced the beam (see Section~\ref{sec:dune:beam} for details).
In terms of terminology, the enhanced neutrino beam is produced using the Forward Horn Current (FHC) polarity while the the enhanced antineutrino beam is produced using the Reverse Horn Current (RHC) polarity.

DUNE's proposed oscillation physics milestones are shown in Table~\ref{tab:physicsMilestones} along with their estimated expected running times.
DUNE's ability to meet these goals depends in part on the large number of neutrino interactions it will be able to observe.
These high statistics derive from the large fiducial volume of the far detector (40~kt with all modules) and the high beam power (1.2~MW initially with later planned upgrades to 2.4~MW).
Table~\ref{tab:statistics} shows the number of expected neutrino interactions in the far detector for 3.5 years of staged running.

\begin{table}
  \caption{Table from~\cite{tdrVol2} showing the expected DUNE oscillation physics milestones and their required exposure in years, assuming the true neutrino mass hierarchy is normally ordered and equal running in with both forward and reverse horn currents. The best fit parameters in~\cite{nufit4} are used.}
  \label{tab:physicsMilestones}
  \centering
  \begin{tabular}{c c}
    \hline
    Physics milestone & Exposure (staged years) \\
    \hline
    $5\sigma$ mass ordering ($\delta_{CP} = -\pi/2$) & 1 \\
    $5\sigma$ mass ordering (all $\delta_{CP}$ values) & 2 \\
    $3\sigma$ CP violation ($\delta_{CP}=-\pi/2$) & 3 \\
    $3\sigma$ CP violation (50\% of $\delta_{CP}$ values) & 5 \\
    $5\sigma$ CP violation ($\delta_{CP}=-\pi/2$) & 7 \\
    $5\sigma$ CP violation (50\% of $\delta_{CP}$ values) & 10 \\
    $3\sigma$ CP violation (75\% of $\delta_{CP}$ values) & 13 \\
    $\delta_{CP}$ resolution of $10^{\circ}$ ($\delta_{CP}=0$) & 8 \\
    $\delta_{CP}$ resolution of $20^{\circ}$ ($\delta_{CP}=-\pi/2$) & 12 \\
    $\text{sin}^{2}2\theta_{13}$ resolution of 0.004 & 15 \\
    \hline
  \end{tabular}
\end{table}

\begin{table}
  \caption{Expected numbers of signal events for the DUNE far detector in the neutrino energy range $0.5-0.8$~GeV for 3.5 years of staged exposures for both FHC and RHC running. The rates shown assume the mass hierarchy is normally ordered, that $\delta_{CP}=0$ and that all other oscillation parameters are at their best fit points in~\cite{nufit4}. Taken from~\cite{tdrVol2}.}
  \label{tab:statistics}
  \centering
  \begin{tabular}{c c}
    \hline
    Signal & Expected events \\
    \hline
    \hline
    Neutrino mode & \\
    $\nu_{\mu}$ & 6200 \\
    $\bar{\nu}_{\mu}$ & 389 \\
    $\nu_{e}$ & 1092 \\
    $\bar{\nu}_{e}$ & 18 \\
    \hline
    Antineutrino mode & \\
    $\nu_{\mu}$ & 1129 \\
    $\bar{\nu}_{\mu}$ & 2303 \\
    $\nu_{e}$ & 76 \\
    $\bar{\nu}_{e}$ & 224 \\
    \hline
  \end{tabular}
\end{table}

% Supernova neutrinos

% Baryon number violation

% Quick bit on BSM physics

\section{Far detector}
\label{sec:dune:fd}
% LAr TPC concept

% Explanation of single and dual phase concepts

\blindtext

\section{Near detector}
\label{sec:dune:nd}

\blindtext

\section{Beam}
\label{sec:dune:beam}

\blindtext
