\chapter{The Deep Underground Neutrino Experiment}
\label{ch:dune}

The Deep Underground Neutrino Experiment (DUNE) is a long-baseline neutrino oscillation experiment currently under construction at two sites in the USA~(\cite{tdrVol1}, \cite{tdrVol2}, \cite{tdrVol3}, \cite{tdrVol4}).
The near detector site which will host the beam production facilities and the multi-component near detector is located at Fermilab, Illinois while the far detector site will be located at Sanford Underground Research Facility (SURF), South Dakota.
This will provide a baseline of 1300~km.

In this chapter, the DUNE detector will be discussed.
Section~\ref{sec:dune:overview} will provide an overview of the detector, its motivation and its physics goals.
Section~\ref{sec:dune:fd} will discuss the design of the far detector.
Section~\ref{sec:dune:nd} will discuss the design of the near detector and Section~\ref{sec:dune:beam} will discuss the design of the beam.

\section{Overview}
\label{sec:dune:overview}

\section{Far detector}
\label{sec:dune:fd}

\blindtext

\section{Near detector}
\label{sec:dune:nd}

\blindtext

\section{Beam}
\label{sec:dune:beam}

\blindtext
